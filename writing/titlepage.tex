\documentclass[12pt, notitlepage]{article}
\usepackage[margin = 1in]{geometry}
\usepackage[english]{babel}
\usepackage[utf8]{inputenc}
\usepackage[table]{xcolor}
\usepackage{graphicx, booktabs, tikz, enumitem, dcolumn, pdfpages, csquotes}
\usepackage[font=normalsize]{caption}%,labelfont=bf
\usepackage{subcaption}

\usepackage{xr-hyper}
\usepackage[colorlinks = TRUE, allcolors = blue]{hyperref}
\externaldocument{appendix}

\renewcommand*\rmdefault{ppl}

\usepackage{setspace}
\setstretch{1.5}

\usepackage[]{titlesec}
    \titleformat*{\section}{\large\bf}
    \titleformat*{\subsection}{\normalsize\it}

% Bibliography
\usepackage[round]{natbib}
\setlength{\bibsep}{5pt}

\widowpenalty=10000
\clubpenalty=10000

\title{\Large Violence, co-optation, and postwar voting in Guatemala}
\author{Francisco Villamil\footnote{Contact: francisco.villamil@uc3m.es}\\\textit{Carlos III-Juan March Institute, Universidad Carlos III de Madrid}}
\date{Word count: 10,922}

% \usepackage[none]{hyphenat}

\begin{document}

\maketitle
\thispagestyle{empty}

\vspace{30pt}

\begin{abstract}
\setstretch{1.1}

Wartime civilian victimization produces a counter-reaction against the perpetrator. However, this effect hinges on the creation of collective memories of wartime events. In many countries, former fighting actors and political elites try to redirect memories of wartime events through denial, propaganda, and co-optation. Previous works have ignored these aspects. I argue that the effect of violence is conditional on the capacity of local communities to build collective memories and bypass those efforts. I test this argument using local-level data from Guatemala. Results show that the effects of state violence on postwar voting depend on prewar exposure to political mobilization.

\end{abstract}

\end{document}
