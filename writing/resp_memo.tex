\documentclass[12pt, a4paper, notitlepage]{article}
\usepackage[margin=2.25cm]{geometry}
\usepackage[english]{babel}
\usepackage[utf8]{inputenc}
\usepackage{csquotes, xcolor}
\usepackage{setspace}
\setstretch{1.3}

% Bibliography
\usepackage[round]{natbib}
\setlength{\bibsep}{5pt}

\usepackage{xr-hyper}
\usepackage[colorlinks = TRUE, allcolors = blue]{hyperref}
\externaldocument{main}
\externaldocument{appendix}

\renewcommand*\rmdefault{ppl}

\usepackage[]{titlesec}
\titleformat*{\section}{\large\bfseries}

\title{\large \textbf{Revision memo}\\{\normalsize Manuscript `Violence, co-optation, and postwar voting in Guatemala', submitted to Conflict Management and Peace Science (CMPS-21-0042.R1)}}
\author{}
\date{\today}

\definecolor{darkgrey}{rgb}{0.5, 0.5, 0.5}
\renewcommand{\mkbegdispquote}[2]{\color{darkgray}}

\begin{document}

\maketitle

% \newpage
\section*{Response to Editor}

\newpage
\section*{Reviewer 4}

First, I would like to thank the reviewer for all the comments. They were very thoughtful and useful in improving the manuscript, particularly in terms of making the argument and the contribution more clear. I respond to each one below.

\noindent\textbf{Response:}

\vspace{15pt}
\noindent\textbf{Comment 1:}
\begin{displayquote}
It is unclear whether the author is focusing on violence or propaganda-while these two things are closely related they are also very different. More clarity on this is needed throughout.
\end{displayquote}

\noindent\textbf{Response:}

\vspace{15pt}
\noindent\textbf{Comment 2:}
\begin{displayquote}
Author states previous work treats relationship between violence and civilian political preferences as a black box - can they expand? What information is missing from previous research?
\end{displayquote}

\noindent\textbf{Response:}

\vspace{15pt}
\noindent\textbf{Comment 3:}
\begin{displayquote}
"Individuals are assumed to objectively interpret violent events, and no attention is paid to those external factors that might influence this interpretation, such as propaganda or cooptation efforts by the perpetrator". -- this sentence is key-would put it in the introduction to clarify contribution from the beginning.
\end{displayquote}

\noindent\textbf{Response:}

\vspace{15pt}
\noindent\textbf{Comment 4:}
\begin{displayquote}
"I focus on the case of Guatemala, which is a good example of the use of denying and cooptation strategies". - does this mean it's unique? How does it compare to other cases?
\end{displayquote}

\noindent\textbf{Response:}

\vspace{15pt}
\noindent\textbf{Comment 5:}
\begin{displayquote}
How does the article relate to research on misinformation? Are there parallels to be drawn from the use of misinformation in political campaigns/where it is most successful?
\end{displayquote}

\noindent\textbf{Response:}

\vspace{15pt}
\noindent\textbf{Comment 6:}
\begin{displayquote}
I'm unclear on how a scorched earth campaign could gain support from the civilian population
\end{displayquote}

\noindent\textbf{Response:}

\vspace{15pt}
\noindent\textbf{Comment 7:}
\begin{displayquote}
With the PAC system-couldn't civilians just go through the motions without actually believing rebels were the enemy? What did the government do to actually convince them?
\end{displayquote}

\noindent\textbf{Response:}

\vspace{15pt}
\noindent\textbf{Comment 8:}
\begin{displayquote}
I still don't understand how the state managed to avoid being blamed for acts of wartime victimization. Need more details in the theory and evidence section on how propaganda could overcome experiences with wartime victimization.
\end{displayquote}

\noindent\textbf{Response:}

\vspace{15pt}
\noindent\textbf{Comment 9:}
\begin{displayquote}
The author needs to differentiate between communities' ability to resist co-optation strategies, and just being ideologically opposed to the government. Isn't it more plausible that these communities were simply ideologically opposed to the government? I understand that these two mechanisms could operate in conjunction with one another but I remain unconvinced that the author's plays a significant role.
\end{displayquote}

\noindent\textbf{Response:}

\vspace{15pt}
\noindent\textbf{Comment 10:}
\begin{displayquote}
Why is there no data on vote share for the FRG in 2011? Why does vote share drop so significantly for both parties over the years?
\end{displayquote}

\noindent\textbf{Response:}

\vspace{15pt}
\noindent\textbf{Comment 11:}
\begin{displayquote}
The author blends together a lot of concepts in their explanation of what determines communities' reaction to propaganda-pre-war political mobilization, literacy, collective story telling, ideological capital-it becomes confusing as to which one is the most important. If the author thinks they are all equally important, that should be laid out from the beginning and in the theory section.
\end{displayquote}

\noindent\textbf{Response:}

\vspace{15pt}
\noindent\textbf{Comment 12:}
\begin{displayquote}
The author should explain why they include the controls that they do-as of right now there is no explanation, just a list of variables.
\end{displayquote}

\noindent\textbf{Response:}

\vspace{15pt}
\noindent\textbf{Comment 13:}
\begin{displayquote}
The paper would benefit from more discussion of the quantitative results-what do they actually mean, what can we conclude from them?
\end{displayquote}

\noindent\textbf{Response:}

\vspace{15pt}
\noindent\textbf{Comment 14:}
\begin{displayquote}
The qualititave section seems ad hoc-what strategy did the author use to find these sources? What sources were reviewed?
\end{displayquote}

\noindent\textbf{Response:}

\newpage
\bibliographystyle{jpr}
\bibliography{REF}

\end{document}
