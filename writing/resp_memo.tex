\documentclass[12pt, a4paper, notitlepage]{article}
\usepackage[margin=2.25cm]{geometry}
\usepackage[english]{babel}
\usepackage[utf8]{inputenc}
\usepackage{csquotes, xcolor}
\usepackage{setspace}
\setstretch{1.3}

% Bibliography
\usepackage[round]{natbib}
\setlength{\bibsep}{5pt}

\usepackage{xr-hyper}
\usepackage[colorlinks = TRUE, allcolors = blue]{hyperref}
\externaldocument{main}
\externaldocument{appendix}

\renewcommand*\rmdefault{ppl}

\usepackage[]{titlesec}
\titleformat*{\section}{\large\bfseries}

\title{\large \textbf{Revision memo}\\{\large Manuscript CMPS-21-0042.R1\\`Violence, co-optation, and postwar voting in Guatemala'}}
\author{}
\date{\today}

\definecolor{darkgrey}{rgb}{0.5, 0.5, 0.5}
\renewcommand{\mkbegdispquote}[2]{\color{darkgray}}

\begin{document}

\maketitle

% \newpage
\section*{Response to Editor}

I would like to thank you for the opportunity to revise the manuscript. As you can see below, I have made several changes to the manuscript in response to each one of the reviewer's comments. In general terms, following the main points raised by the reviewer:

\begin{itemize}
  \item I have rewritten some parts of the Introduction, trying to state more clearly the research question of the manuscript and the main contribution made, as per some comments by the reviewer about how the focus of the paper was not too clear. I have also justified the use of the Guatemalan case and stated how it relates to other cases around the world.
  \item I now specify further what is missing in the previous literature and how this paper in particular contributes in filling this gap. I have also included a reference to the misinformation literature, per the reviewer's suggestion.
  \item I have rewritten some parts of the theory, improving conceptual clarity and trying to connect it better to the research question stated in the Introduction, in addition to clarifying some of the comments made by the reviewer, such as the possibility that civilians could go through the PAC system without actually experiencing a belief change.
  \item I have added some general discussion to both the quantitative results and the qualitative sources, so it should now be more clear the connection and complementarities among them.
\end{itemize}

I hope you find that the manuscript has improved in this new version. In particular, I hope that it is now easier to read, that the contribution and research question are more clear, and that the advantages and limitations of the overall empirical evidence are more transparent.


\newpage
\section*{Reviewer 4}

First, I would like to thank the reviewer for all the comments. They were very thoughtful and useful in improving the manuscript, particularly in terms of making the argument and the contribution more clear. I respond to each of them below.

\vspace{15pt}
\noindent\textbf{Comment 1:}
\begin{displayquote}
It is unclear whether the author is focusing on violence or propaganda-while these two things are closely related they are also very different. More clarity on this is needed throughout.
\end{displayquote}

\noindent\textbf{Response:} I thank the reviewer for this comment. The focus of the manuscript is on the effect of wartime violence and how this effect changes depending on the local exposure to propaganda--or the local capacity to resist propaganda and co-optation efforts. However, it is true that the manuscript can be a bit confusing about its main contribution.

To improve this, I have rewritten some parts of it to make this contribution more clear, particularly in the Introduction, where I try to highlight the research question and have also included the suggestion made in comment 3 below. Now the relevant paragraph in page 1 reads as:


``How do these strategies alter the effect of wartime victimization on political preferences after the war?
The theoretical framework found in previous research is not able to answer this question, as past works do not usually account for any local factor that could mediate the long-term effect of civilian victimization.
Individuals are assumed to objectively interpret violent events, and no attention is paid to those external factors that might influence this interpretation, such as propaganda or co-optation efforts by the perpetrator.

This paper tries to address this limitation, studying the effect of violence on postwar preferences and how this effect can vary depending on the exposure and reactions to propaganda and co-optation strategies.''

I hope these changes help to clarify what the manuscript is about and what is its main contribution.

\vspace{15pt}
\noindent\textbf{Comment 2:}
\begin{displayquote}
Author states previous work treats relationship between violence and civilian political preferences as a black box - can they expand? What information is missing from previous research?
\end{displayquote}

\noindent\textbf{Response:} I thank the reviewer for this comment. I agree that the previous version of the manuscript was not specific enough about the gaps in previous research. What I tried to convene here is that the effect of violence is usually theorized to take place homogenously, without discussing potential third-factors that influence it. In order to make this more clear, I have added the following lines to the current version of the manuscript (page 4):

``Previous research has treated the process that leads from violent events to a change in political preferences as a black box.
In other words, we know little about potential heterogenous effects or the factors that explain why legacies of violence might be present in some cases but not in others.''


\vspace{15pt}
\noindent\textbf{Comment 3:}
\begin{displayquote}
"Individuals are assumed to objectively interpret violent events, and no attention is paid to those external factors that might influence this interpretation, such as propaganda or cooptation efforts by the perpetrator". -- this sentence is key-would put it in the introduction to clarify contribution from the beginning.
\end{displayquote}

\noindent\textbf{Response:} I thank the author for this suggestion, which I think has benefitted the manuscript in making the contribution more clear. As I say in the response to comment 1, I have moved this sentence to the Introduction and placed it in a paragraph where I state the main research question.

\vspace{15pt}
\noindent\textbf{Comment 4:}
\begin{displayquote}
"I focus on the case of Guatemala, which is a good example of the use of denying and cooptation strategies". - does this mean it's unique? How does it compare to other cases?
\end{displayquote}

\noindent\textbf{Response:} I thank for the author for this comment, which points to the external validity of this study. I agree that this is a crucial point that needs to be justified.

As I mention in the manuscript, Guatemala is a good case example of postwar efforts to reframe conflict memories, but this does not mean it is unique. There are many countries where political actors, usually the winning side, take on an active role spreading propaganda and misinformation and co-opting local leaders, in order to stop the opposite side from gaining support as a consequence of wartime violence and to cement their own power. Spain and Sri Lanka are two good examples of these strategies, mentioned in the manuscript. In Sri Lanka, for example, the government actively tried to blame civilian deaths on the LTTE, monopolised the memorialisation of the war through the lens of Sinhala nationalism, co-opted Tamils to gather information about local preferences, and dismantled symbols somewhat related to Tamil nationalism, among other things \citep{Subramanian:2016vk, Seoighe:2017aa, Kapur:2020we}.
And even though an active role of state authorities in steering collective memories might not be present in every postwar context, narrative conflicts are probably present virtually in all conflicts, given the uncertainty that permeates all civil wars.
Thus, the argument developed and tested here in Guatemala should travel to any postwar context where there is a narrative conflict and where political actors are trying to win it, at least in the sense that the consequences of wartime violence will depend on these propaganda and co-optation strategies and on the capacity of local communities to face them.

In order to try to make this point more clear, I have rewritten the paragraph in the Introduction where I introduce the case of Guatemala:

``I focus on the case of Guatemala, where there was an active debate about the responsibility of wartime events and where state authorities actively engaged in cooptation strategies as part of the counterinsurgency campaign.
In the aftermath of the victimization campaign that took place in Guatemala in the early 1980s, the state tried to deny war events, set up a discourse that justified military actions as a necessary step to bring peace to the country, and forcefully recruited civilians into paramilitary units of local defense.
This example is far from unique, as in many postwar contexts there is a controversy about blame attribution, where active political actors take an active role in trying to define the mainstream discourse.
The fact that a civil war is usually a high-uncertainty context makes it an ideal space for rumors and propaganda to spread \citep{Schon:2021wf}, a situation that in many cases strongly defines the postwar context.''

\vspace{15pt}
\noindent\textbf{Comment 5:}
\begin{displayquote}
How does the article relate to research on misinformation? Are there parallels to be drawn from the use of misinformation in political campaigns/where it is most successful?
\end{displayquote}

\noindent\textbf{Response:} I thank the reviewer for this thoughtful question. I agree that the discussion on propaganda and co-optation might be related to the issue of misinformation during political campaigns. However, it is also true that it is a bit difficult to connect this discussion on propaganda in the context of an armed conflict in an underdeveloped country with the literature on misinformation, which focuses heavily on American politics or other developed democracies.

That said, I agree with the reviewer that it is important to mention these works. In particular, I consider that the closest concept in the misinformation literature to what is being discussed in this manuscript is that of rumors, which is a different phenomenon than misinformation. I have included a reference to \citet[243]{Berinsky:2017ty} in the manuscript, where rumors are defined as ``fringe beliefs ... [that] acquire their power through widespread social transmission,'' linking this idea with the discussion on the formation of collective memories in a civil war context.

\vspace{15pt}
\noindent\textbf{Comment 6:}
\begin{displayquote}
I'm unclear on how a scorched earth campaign could gain support from the civilian population
\end{displayquote}

\noindent\textbf{Response:} I thank the reviewer for raising this point, which points to a crucial part of the manuscript. I agree that this is a counterintuitive claim, but highlighting how violence can have heterogenous effects is precisely the goal of the manuscript.
A short answer is that it was mainly through fear that the Guatemalan state could have gained local cooperation and even support through violence. In other words, in a context where there are high levels of uncertainty and low security, being perceived as the strongest side could have had this effect, which would have had long-term consequences in the postwar period. In any case, previous research has shown that this type of effects can take place, at least while the conflict lasts \citep{Schubiger:2021aa}.

In terms of the manuscript, paragraph 2 in page 2 discusses the role of fear in justifying such a potential outcome:

``In the aftermath of the victimization campaign that took place in the early 1980s, the Guatemalan state tried to deny war events, set up a discourse that justified military actions as a necessary step to bring peace to the country, and forcefully recruited civilians into paramilitary units of local defense.
In a country where many areas had been isolated from national politics and where political illiteracy was rampant, this strategy was to a large extent successful.''

Similarly, in page 7, I discuss how the state could have been successful in doing so, through the propaganda and co-optation strategies that are discuss throughout the text:

``The victimization campaign of the early 1980s disrupted local communities. It pitted neighbors against neighbors and installed a climate of fear and distrust that would leave long-term consequences \citep{Burrell:2013aa}.
The government successfully destroyed Maya social organizations and gained forced cooperation from local people. (...)

The PAC system was designed to ensure compliance within the countryside.
It forced local civilians to `defend' themselves from the guerrilla and, in many cases, successfully convinced them that they were on the good side and that violence and insecurity was the rebels' fault.''

I hope that, after changing the way the research question and the framing of the contribution are presented, it is now more clear how state violence could have had such a counterintuitive effect. As I acknowledge in the discussion of the quantitative results (see response to comment 13 below), the increase in support to the perpetrator as a result of violence is not directly supported by the data, even though we do not know the support they would have had in the absence of violence, and it is an aspect I discuss in the qualitative section.

\vspace{15pt}
\noindent\textbf{Comment 7:}
\begin{displayquote}
With the PAC system-couldn't civilians just go through the motions without actually believing rebels were the enemy? What did the government do to actually convince them?
\end{displayquote}

\noindent\textbf{Response:} I thank the reviewer for this comment. I agree that this is a real possibility which should be at least accounted for. In any case, even though civilians could just participate without believing the state's discourse, this was precisely the goal of the state. The PAC system was designed not only with the aim of improving security (in a counter-insurgency sense) but also to ensure compliance. I have slightly changed the paragraph where I discuss the PAC system (page 8) to account for this, highlighting how it arguably managed to do so by transforming local social structures:

``The system aimed to destroy community networks and erode local interpersonal trust, which presumably had helped the guerrillas organize a local base of support \citep{SaenzdeTejada:2004aa}.
Although civilians could in theory take part in the PAC without changing their beliefs, the system was designed with the goal of doing so.
As \citet[641]{Bateson:2017aa} says, ``during the Guatemalan civil war, the military made a concerted effort to socialize and re-educate the civilians of the Western Highlands.''''


\vspace{15pt}
\noindent\textbf{Comment 8:}
\begin{displayquote}
I still don't understand how the state managed to avoid being blamed for acts of wartime victimization. Need more details in the theory and evidence section on how propaganda could overcome experiences with wartime victimization.
\end{displayquote}

\noindent\textbf{Response:} I thank again the reviewer for this comment which, together with comment 6 above, points to the credibility of this central claim in the theoretical argument of the manuscript, which states that violence has heterogenous effects, and does not always backfire in the form of countermobilization. In this case, there is a small different: gaining support as a result of victimization is not the same as avoiding blame. Avoiding being blamed for wartime victimization is probably a common phenomenon in many wartime contexts, as in the examples of Spain and Sri Lanka reviewed above and mentioned in the manuscript. For blaming to take place, individuals need to learn about who was responsible for wartime events and need to develop a narrative of the conflict that is coherent with this outcome. Efforts such as the Sri Lankan government's discourse that the blame for the violence was on the `terrorists' (i.e. the LTTE) or the idea, supported by the Franco's regime in postwar Spain, that violence against civilians had no political reasons but personal ones are some of the strategies used to accomplish this. And considering the lack of information and the pressure civilians face during a war and a postwar period, it is not so surprising that they are sometimes successful.

In addition to the changes mentioned in the response to comment 6, I have also added some clarifications at the end of page 8, mentioning the role of fear and co-optation mechanisms in explaining this outcome. The final paragraph in this section (pages 8--9) summarizes this idea:

``I argue that the strategy carried out by state authorities was key to understand the long-term consequences of victimization in Guatemala.
First, this strategy was in many cases successful in establishing a narrative favoring the state's discourse.
In these areas, the state managed to avoid being blamed for acts of wartime victimization, through a fear-based mechanism and the co-optation of local communities, inhibiting the creation of collective memories.
On the contrary, some communities were more capable of resisting state propaganda and building memories of the conflict which led to long-term counter-mobilization as a response to violence.
My argument states that this resistance was possible in places that had been exposed to prewar leftist mobilization, as they had the ideological tools to interpret wartime events differently.''

I hope that it is now more clear how state violence did not always have the expected backfiring effect, which is the core of the theoretical argument.

\vspace{15pt}
\noindent\textbf{Comment 9:}
\begin{displayquote}
The author needs to differentiate between communities' ability to resist co-optation strategies, and just being ideologically opposed to the government. Isn't it more plausible that these communities were simply ideologically opposed to the government? I understand that these two mechanisms could operate in conjunction with one another but I remain unconvinced that the author's plays a significant role.
\end{displayquote}

\noindent\textbf{Response:} I thank the reviewer for this comment, which raises a crucial issue. In particular, there are reasons to think that the outcomes was a product of just left-leaning preferences, rather than the local ideological context enabling local communities to resist propaganda and build collective memories. This mechanism means that propaganda and co-optation would not have any role.

I agree that this is a real possibility, but I think this mechanism is more of a complementary channel rather than an alternative to the main argument. It is clear that left-leaning communities would have judged wartime events differently, but that does not mean that the state's campaign to spread false information and co-opt local communities did not play any role in defining postwar preferences. And although it is true that the quantitative results offer limited evidence on this, I do review some of these points in the section where I present the qualitative evidence.

I discuss all these points in the main text, at the end of the theory section in pages 10--11:

``An alternative mechanism would be that some local communities reacted differently to state violence not because they were less sensitive to propaganda and co-optation but because they had leftist preferences to start with.
Previous research shows that the short-term effect of wartime violence on combatant support varies depending on whether the perpetrator is part of the in-group or not \citep{Lyall:2013aa}, as civilians evaluate wartime events according to their previous political orientation \citep{Silverman:2019aa, Pechenkina:2020ul}.
Left-leaning communities would judge wartime events and assign blame according to their preferences.
Thus, even without taking into account propaganda and co-optation strategies by the state, areas that had been more exposed to prewar leftist mobilization should judged wartime state actions more negatively.

This mechanism hinges on an ideology-driven active judgment of state violence, but it is not at odds with the former discussion on the capacity of some communities to resist state propaganda, which is also driven by local ideology.
During the civil war, information about wartime events in Guatemala was likely very limited, so beliefs about wartime events were formed in a context of high uncertainty.
Along these lines, recent research suggests that rumor evaluation in wartime contexts is largely determined by surrounding social networks \citep{Schon:2021wf}.
Active judgment of wartime events is thus very much intertwined with the `narrative contests' that unfold in a situation where state actors are actively trying to spread misinformation about what happened during the war.
The key aspect here is whether propaganda and co-optation had an effect on those communities that did not react against violence.

The quantitative evidence can only offer limited evidence about these mechanisms.
However, in a separate section below, I discuss qualitative evidence in support of the assumptions of the argument and the proxy variables used, including the role of state propaganda.''


\vspace{15pt}
\noindent\textbf{Comment 10:}
\begin{displayquote}
Why is there no data on vote share for the FRG in 2011? Why does vote share drop so significantly for both parties over the years?
\end{displayquote}

\noindent\textbf{Response:} The reason why there is no data for the FRG in 2011 is that this party did not participate in national elections during this year. Indeed, the party was rebranded as the Institutional Republican Party (PRI) after that year.

Regarding the drop in vote share for both the URNG and FRG over the years, this is a very relevant question that was also raised by other reviewers. One of the potential reasons is that the war cleavage was not as significant as democratic rule was established in Guatemala. Indeed, left-wing parties did not show any discourse that pointed to the revolutionary left after 2007 \citep{Ibarra:2008to}.

In any case, this issue raises some concerns regarding the quantitative analyses. The way I deal with this is by including additional results in Appendix F with the results of the main models but including more parties on both the right and the left as the dependent variable. Thus, I try to capture electoral support that was ideologically coherent with the URNG and FRG but went to different parties. In particular, together with the FRG, I include the Partido Patriota (PP) and the Frente de Convergencia Nacional (FCN).
Both PP and FCN are major right-wing parties that have displayed a discourse similar to the FRG's and have been linked to military officers accused of having committed human rights violations during the civil war.
On the left, I include the Unidad Nacional de la Esperanza (UNE) together with the URNG. The UNE is a mainstream left-wing party that was headed by Álvaro Colom, former President of Guatemala and former member of the URNG.

\vspace{15pt}
\noindent\textbf{Comment 11:}
\begin{displayquote}
The author blends together a lot of concepts in their explanation of what determines communities' reaction to propaganda-pre-war political mobilization, literacy, collective story telling, ideological capital-it becomes confusing as to which one is the most important. If the author thinks they are all equally important, that should be laid out from the beginning and in the theory section.
\end{displayquote}

\noindent\textbf{Response:} I thank again the author for this comment. I agree that the manuscript would benefit from some conceptual clarity and a more specific use of some of the terms. While they do not all refer to the exact same thing (prewar political mobilization is what determines collective story telling, for instance), it is true that some of them overlap enough to be redundant.

In order to increase the clarity of the manuscript, I have rewritten some parts of the manuscript and changed the use of some terms. For instance, I have removed all references to `ideological capital', which did not add much to the discussion. Instead of these terms, I now try to focus more on the actual mechanism and highlight how prewar political mobilization by the left gave some communities the tools to build collective memories of the war, thus resisting state propaganda. I hope the new version is easier to follow and the theoretical framework is not conceptually confusing.

\vspace{15pt}
\noindent\textbf{Comment 12:}
\begin{displayquote}
The author should explain why they include the controls that they do-as of right now there is no explanation, just a list of variables.
\end{displayquote}

\noindent\textbf{Response:} I thank the reviewer for this comment. I have now included brief explanations for each control variable, and the paragraph where I discuss them (pages 17--18) now reads as:

``Using data from the 1973 census, obtained from the replication data for \citet{Sullivan:2012aa}, I include the logged population of each municipality, the share of indigenous population, and the literacy rate, which were important drivers of both wartime violence and postwar electoral results.
To control for geographic factors that determined wartime dynamics, I include two measures of terrain ruggedness: the standard deviation of elevation within each municipality \citep{Mapzen:2018aa} and the share of forest cover from the GlobCover Land Cover Maps \citep{Arino:2012aa}.
I include a measure of rebel violence before 1978, the logged number of killings for every 1000 inhabitants following the CIIDH dataset \citep{Ball:1999aa}, to control for early rebel presence.
I also include the logged distance (km) from each municipality to Guatemala City and the logged area of each municipality (in $km^2$), to control for the isolation mechanisms related to both wartime violence and electoral behavior.''

I hope these changes help to understand better the justification for each control variable, even though a more detailed discussion is complicated given space constrains.

\vspace{15pt}
\noindent\textbf{Comment 13:}
\begin{displayquote}
The paper would benefit from more discussion of the quantitative results-what do they actually mean, what can we conclude from them?
\end{displayquote}

\noindent\textbf{Response:} I agree with the reviewer that a concluding discussion of the quantitative results would benefit the manuscript, particularly in highlight what parts of the argument find support in the results and which ones do not. In response, I have added the following paragraph at the end of the `Results' section, wrapping up what can be learned from the quantitative analyses and what it being address with additional qualitative evidence (page 24):

``Overall, the results show evidence that violence only backfired in municipalities that were allegedly exposed to prewar leftist mobilization.
The lack of support for the last hypothesis means that it unlikely that state propaganda and co-optation was successful in increasing support as a result of victimization, but this does not mean that it was successful in avoiding counter-mobilization.
As discussed above, the quantitative analyses can offer only limited evidence on the mechanisms.
In the next section below, I offer qualitative evidence on several steps of the mechanism, including the role of state propaganda in preventing victimization from backfiring.''

\vspace{15pt}
\noindent\textbf{Comment 14:}
\begin{displayquote}
The qualititave section seems ad hoc-what strategy did the author use to find these sources? What sources were reviewed?
\end{displayquote}


\noindent\textbf{Response:} I thank the reviewer for this comments. I agree that this aspect is fundamental in justifying the qualitative section and it was not specified in the manuscript. The qualitative evidence shown in the last section, which tries to provide evidence for the mechanism and some aspects of the theory that are not tested with the empirical analyses, comes from a variety of sources, including historical archives (such as the newspaper articles) and secondary historiography, particularly from anthropological research. However, there was no systematic strategy to find these sources, other than looking for previous works on the topic. The newspaper articles come from archival work done at the CIRMA (\textit{Centro de Investigaciones Regionales de Mesoamérica}) in Antigua, Guatemala, in April 2018.

In any case, this section can be thought of as a series of `weak' tests of several aspects of the theoretical argument, in the sense that none of them confirms or falsifies the argument. Thus, they are close to what has been called in the literature on process tracing as `straw in the wind' tests \citep{Collier:2011ve}.

In order to make this more explicit, and even though space constrains do not allow me to discuss this more in detail, I have included the following lines to the manuscript, at the beggining of that section (page 25):

``The qualitative evidence provided here comes from a mix of sources, including anthropological studies and archives, but does not constitute a strict test of the process.
Rather, it offers a series of non-decisive, `straw in the wind' tests \citep{Collier:2011ve} for several steps of the mechanism and the proxy variables used that cannot be directly confirmed in the quantitative analyses.''

\bibliographystyle{jpr}
\bibliography{REF}

\end{document}
