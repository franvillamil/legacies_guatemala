\documentclass[12pt, a4paper, notitlepage]{article}
\usepackage[margin=2.25cm]{geometry}
\usepackage[english]{babel}
\usepackage[utf8]{inputenc}
\usepackage{csquotes, xcolor}
\usepackage{setspace}
\setstretch{1.3}

% Bibliography
\usepackage[round]{natbib}
\setlength{\bibsep}{5pt}

\usepackage{xr-hyper}
\usepackage[colorlinks = TRUE, allcolors = blue]{hyperref}
\externaldocument{main}
\externaldocument{appendix}

\renewcommand*\rmdefault{ppl}

\usepackage[]{titlesec}
\titleformat*{\section}{\large\bfseries}

\title{\Large \textbf{Revision memo}\\{\large Manuscript `Violence, co-optation, and postwar voting in Guatemala', submitted to Conflict Management and Peace Science (CMPS-21-0042)}}
\author{}
\date{\today}

\definecolor{darkgrey}{rgb}{0.5, 0.5, 0.5}
\renewcommand{\mkbegdispquote}[2]{\color{darkgray}}

\begin{document}

\maketitle

% \newpage
\section*{Response to Editor}

I would like to thank you for the opportunity to revise the manuscript. As you will see, I have made thorough revisions, seriously considering each point raised by the reviewers.

In general terms, I have made changes to both the theory and the empirics. The main two points raised by the reviewers are related to the lack of a strong connection between the theoretical framework and the empirics and to the lack of qualitative evidence supporting the mechanism. I have introduced two main changes to account for these points: a) the theoretical framework has been rewritten, highlighting that an alternative mechanism can also explain the results without being at odds with the main theory, and b) I have introduced a new section at the end of the manuscript (`Identifying the mechanism', pages 24--28) including qualitative evidence for each of the steps of the proposed mechanism, including the assumed relationship between road accessibility and prewar exposure to leftist political mobilization.

In addition, I have also introduced several other changes, including a more thorough Appendix with some of the tests and changes suggested by the reviewers (e.g. include more parties on the right and the left, full tables, accounting for turnout, etc), a rewriting of some parts of the text, a better discussion of the limitations, further references on the Guatemalan context, and various clarifications (e.g. identifying the model and variable value in predicted probability plots, including full result tables in the Appendix, etc).

I hope you will find the manuscript to have significantly improved through this process.
Below I address each of the reviewers' comments in turn.

\newpage
\section*{Reviewer 1}

First, I would like to thank the reviewer for all the comments. They were very thoughtful and useful in improving the manuscript. I respond to each one below.

\vspace{15pt}
\noindent\textbf{Comment 1:}
\begin{displayquote}
My biggest hesitation is the author’s use of a proxy in measuring exposure to prewar political mobilization. I completely understand the nature of data limitations and the decision to pursue this route. I’m sure that the author(s) understands that given this decision, much scrutiny will be placed upon the choice of a proxy.
\end{displayquote}

\noindent\textbf{Response:} Thank you for this comment, which does indeed point at a crucial limitation of the manuscript raised by the other reviewers as well.
I discuss the changes to each specific comment more in detail below, but in general terms, the changes I have implemented with regards to this comment are:

\begin{itemize}
  \item I have included a new section at the end (`Identifying the mechanism', pages 24--28) presenting qualitative evidence coherent with the road accessibility assumption (and each of the steps of the mechanism)
  \item I have rewritten some parts of the theoretical framework to acknowledge the existence of alternative mechanisms (page 10)
  \item I have rewritten the discussion of the proxy choices to discuss the alternative explanation based on insularity from national politics (page 17)
\end{itemize}

I hope you find that the manuscript has improved and that this new version better addresses the problem of the proxy choice.

\vspace{15pt}
\noindent\textbf{Comment 2:}
\begin{displayquote}
Currently, I see two alternative, competing theories for your empirical relationship. The first is that the government, given better road-access, is better able to carry out violence in these areas. Thus, we would anticipate greater changes in political preferences following these state-killings. I believe you successfully account for this alternative explanation in Appendix Tables A2 and A3. Although, I do worry about the potential inflation of standard errors due to multicollinearity — not only between your two proxy variables, but also with the other variables, like forest cover, elevation, and distance to capital. I would recommend simpler analyses that isolate these proxy variables. I also question the dependent variables’ form, as its not specified explicitly. Is it still log- transformed? Is it still normalized to population?
\end{displayquote}

\noindent\textbf{Response:} I agree with the reviewer that these issues might be of concern when trying to rule out this alternative explanation, which is indeed crucial for the validity of the results. Following the suggestion above, I now include in the corresponding section (Appendix~\ref{app:lm_violence}) models without all the control variables, so the effects of the two proxies can be compared across the different specifications.
Because of the increase of models, I include four tables for the results of state violence for the whole sample and the reduced sample of most affected departments (tables~\ref{tab:lm_govt_vi} and \ref{tab:lm_govt_vi_hv}), and the same for rebel violence (tables~\ref{tab:lm_rebels_vi} and \ref{tab:lm_rebels_vi_hv}).

Results remain the same. Although the simplest model---without controls or department fixed effects, in the whole sample---does show a positive relationship between the share of non-paved roads and state violence, this effect disappears when department fixed effects are included (even without including any control variable). It also disappears when the model is run in a sample of only the most affected departments.

The first result is not surprising given that wartime activity was concentrated in the western highlands, where road network and terrain ruggedness are worse. Yet, once we compare only within departments, the relationship disappears.
Given that all the main analyses include department fixed effects and that I also test whether the results are also present in the reduced sample of the most affected departments, it should not be a concern for the interpretation of the results.

Finally, as per the reviewer's suggestion, I have specified the dependent variables' form in the same Appendix~\ref{app:lm_violence}: as in the main analyses, state and rebel violence are also log-transformed and normalized to population.

\vspace{15pt}
\noindent\textbf{Comment 3:}
\begin{displayquote}
The second alternative explanation that I feel is not accounted for is that what is potentially being measured by these proxies is actually insularity from national- level politics, which explains why state killings in these less accessible, more remote areas have a muted effect on political preferences. I would assume these are rural communities with lesser access to information — such as radios, TVs, and internet-connected devices. I also acknowledge that the author(s) present qualitative evidence that the government was very much involved in propaganda efforts (on Pages 10-11). But I wonder if the author(s) can account for this alternative explanation in their quantitate analysis. I’m not sure how the author(s) could remedy this concern with their existing data. Perhaps if the authors could account for voter-turnout, or if possible, income levels or technology penetration, I would be fully convinced of their theory.
\end{displayquote}

\noindent\textbf{Response:} I thank the reviewer for this insightful comment, which hints at a very important point. I do agree that insularity from national-level politics as an alternative explanation was not accounted for in the previous version of the manuscript, particularly since not much evidence was provided in support of the role of propaganda. I have made a few changes to better account for this.

First, I include in the section presenting the proxy variables a discussion of how these two variables could also be interpreted as insularity from national politics, instead of a stronger exposure to state propaganda (on page 17). What I try to argue is that even if I emphasize the role of state propaganda, I do not think that this explanation is so much at odds with the current theory, as it was precisely this insularity from national politics which made these communities more vulnerable to state propaganda.

Second, in the new section discussing the qualitative evidence for the mechanism (`Identifying the mechanism', pages 24--28), I present evidence from secondary sources on the role of state propaganda in modifying collective memories of the conflict and how it was precisely more isolated communities the ones that were more vulnerable to these efforts and the ones that did not engage in postwar commemoration activities (pages 26--27).

Finally, following this and other reviewer's suggestions, I have made some changes to the empirical analyses. On the one hand, I now include turnout in the cross-sectional analyses using election-specific samples (Appendix~\ref{app:results_year}). I do not include this control variable in the main analyses because of the high number of missing observations for some election years (particularly in 2007 and 2003). However, tables~\ref{tab:lm_URNG_roads_year} to \ref{tab:lm_FRG_base_year} show that results do not change when including this control variable.
On the other hand, I do include now the literacy rate in the models in the main text which, although it is not directly related to political participation, might be the best proxy available for political isolation.

\vspace{15pt}
\noindent\textbf{Comment 4:}
\begin{displayquote}
Besides accounting for alternative explanations, I believe the author(s) needs to connect their theory more with their choice of a proxy. For example, the author(s) states on Page 18:
\begin{itemize}
  \item[] ``In particular, I assume that accessibility in terms of road infrastructure determined how much exposure local communities had to these external political actors, who expanded throughout the country from the capital and main cities to bring new political ideas and organize the local population.''
\end{itemize}
It is critical that the author(s) substantiates this assumption. As of now, in the “Historical Context” section on Page 8 and “The role of prewar mobilization” subsection on Pages 11-12, the spatial origins of these opposition groups are not exactly clear; did they originate in lesser developed, more remote regions or did they spread outwards from the more developed cities and communities connected to the Pan-American Highway? From my perspective, the latter argument is necessary in explaining your theory. However, it appears that the Catholic Action movement originated in the cities and expanded outwards, while the peasant organizations among the indigenous populations were already established beforehand (note: I don’t have any more background on the Guatemalan conflict than what is presented here).
\end{displayquote}

\noindent\textbf{Response:} I thank the reviewer for this comment. I agree that providing more evidence for this assumption is necessary.

I have included qualitative evidence supporting this relationship in the mechanism section, on pages 24--25. Among other things, I refer to a study by \citet{Esparza:2018uw} of one area in Chupol, in the department of Chichicastenango, where she says that it was precisely in communities close to the Pan-American Highway where the Liberation Theology priests went to more often. I also present evidence that in more isolated communities this process did not take place, at least with the same intensity. Regarding the peasant organizations, the main activists also originated from the main cities, and local organizations usually emerged after external actors arrived, including Catholic Action and foreign priests.

\vspace{15pt}
\noindent\textbf{Comment 5:}
\begin{displayquote}
Finally, given that your theory suggests that the government co-opted civilian populations through propaganda, it’s strange — and somewhat contradictory — that you follow these discussions with this statement on Page 11:
\begin{itemize}
  \item[] ``Matanock \& Garcia-Sanchez (2018) show that civilians falsify their reported support for the military when asked about the counterinsurgency in Colombia, particularly in areas previously held by the insurgents. Preference falsification helps to explain the apparent success of counterinsurgent campaigns, as fear is a major factor explaining the negative effect of repression on opposition activities (Young, 2019).''
\end{itemize}
This suggests co-optation is through fear of repression, not altering public opinion to generate genuine support for the government. It also makes it more unclear how this fear translates to votes, which I presume would not be motivating factor in one’s vote during the democratic elections (at least, without greater context). I recommend omission of this part.
\end{displayquote}

\noindent\textbf{Response:} I thank the reviewer for this comment. Even though fear was indeed part of the mechanism through which the Guatemalan state managed to get local cooperation---as some of the qualitative evidence testifies to, for example---, I agree that this discussion of fear contradicts part of what is being discussed in the section.

I have now removed this paragraph, and include instead at the end of the theoretical section a more lengthy discussion of an alternative mechanism based on an ideologically motivated reaction to state violence (page 10).

\vspace{15pt}
\noindent\textbf{Comment 6:}
\begin{displayquote}
In my view, the largest obstacle to recommending publication is accounting for the alternative explanation of the proxy measures capturing insularity and not the opposition’s political mobilization. I also feel it’s important for the author(s) to contribute more to substantiating their assumptions regarding their proxy variables (i.e., accessibility = prewar political mobilization). My other recommendations I feel are minor and not fatal to the manuscripts advancement. Otherwise, I feel this manuscript makes a worthy contribution to our knowledge of post-war political attitudes. For these reasons, I recommend a Major Revision.
\end{displayquote}

\noindent\textbf{Response:} I thank the reviewer for this and the other comments, which have helped me greatly in improving the manuscript. As explained above, I have tried to address these comments by (a) rewriting the discussion of the variables, and the theoretical framework to a lesser extent, to account for a potential explanation based on isolation from national politics, and (b) including a new section at the end of the manuscript presenting qualitative evidence in support of both the choice of the proxies and each step in the mechanism.

I hope these changes properly address these major points and make the manuscript more coherent and credible.

\vspace{15pt}
\noindent\textbf{Comment 7:}
\begin{displayquote}
Minor Points:
\begin{itemize}
\item I would prefer the control variables to be shown in the main tables. They are not even provided in the appendix — which is suspicious!
\end{itemize}
\end{displayquote}

\noindent\textbf{Response:} I agree with the reviewer that the full tables should be available. I have now placed them in Appendix~\ref{app:results_tablong}, and included a footnote in the main text (page 18) pointing to this section.

\vspace{15pt}
\noindent\textbf{Comment 8:}
\begin{displayquote}
\begin{itemize}
\item For the presentation of predicted probabilities, all figures will require the model used to be specified. Further, given the presence of fixed-effects for departments and elections, I don’t believe its possible to state “All other variables are kept at their mean,” as written on Page 22. Which election year and department were chosen for deriving predicted probabilities? From your appendix results, its quite clear that the hypothesized relationship weakens over time.
\end{itemize}
\end{displayquote}

\noindent\textbf{Response:} I have now specified the model and the specific values for the fixed effect variables in each of the figure captions. In particular, every predicted probability plot was calculated for the Quiché department and 1999 elections. The reviewer is right in saying that the relationship weakens over time (a point also raised by R2), so I have now included a paragraph discussing this finding at the end of the results section (pages 23--24).

\vspace{15pt}
\noindent\textbf{Comment 9:}
\begin{displayquote}
\begin{itemize}
\item A recent article in JCR discusses how road-access is predictive of conflict. The author(s) may find this insightful to their choice of proxy:
\item[] ``Roads to Rule, Roads to Rebel: Relational State Capacity and Conflict in Africa.'' Carl Muller-Crepon, Philipp Hunziker, and Lars-Erik Cederman. Journal of Conflict Resolution (2021).
\end{itemize}
\end{displayquote}

\noindent\textbf{Response:} I thank the reviewer for this reference, which is a very good example of a methodological approach to calculate state capacity from road maps. I was aware of this article, and the reason that I did not take a similar approach in this manuscript is that the goal is slightly different. \citet{Muller-Crepon:2021va} try to develop a measure of relational state capacity, which incorporates both the capacity of the central state authorities and the internal connectedness of an ethnic group. In my case, I am only concerned with how easy was to move around a municipality, but not so much with the capacity of the central state to reach these areas (in that case, the problem of the road proxy measuring insularity from national politics could be much worse). Moreover, calculating shortest paths in a time-weighted road network has an extremely high computational cost, as it involves calculating every existing path between two nodes.

\newpage
\section*{Reviewer 2}

I would like to thank the reviewer for all the comments. They were very thoughtful and useful in improving the manuscript. I respond to each one below.

\vspace{15pt}
\noindent\textbf{Comment 1:}
\begin{displayquote}
Overall, I thought this article had some intriguing empirical findings that could have real value to the literature on the legacy of conflict and violence. As currently constructed, it is however marred by two issues: (1) some substantial slippage between its theory and empirics, and (2) a related lack of qualitative (or quantitative) evidence to nail down the mechanisms at work here. While these issues are not trivial, I think they can perhaps be addressed or at least significantly mitigated and so I think an R\&R opportunity at CMPS would be possible.
\end{displayquote}

\noindent\textbf{Response:} I thank the reviewer for all these detailed comments, which were extremely helpful in improving the manuscript. I discuss each comment more in detail below, but in brief, the new version of the manuscript includes two main changes that account for these two points (disconnection between empirics and theory, and lack of qualitative evidence):

\begin{itemize}
  \item I have rewritten the theoretical framework, including a discussion of a potential alternative mechanism that does not rely on state propaganda but on an ideologically motivated reaction to violence
  \item I have included a new section (`Identifying the mechanism', pages 24--28) after the `Results' section, including qualitative evidence coherent with each of the steps of the mechanism, including the road accesibility assumption
\end{itemize}

I hope that these changes help to improve the credibility and overall quality of the empirical analyses, and of the manuscript in general.

\vspace{15pt}
\noindent\textbf{Comment 2:}
\begin{displayquote}
1) Theory vs. empirics: in a nutshell, the authors argue that violent events can be interpreted in different ways by different groups of people based on their ability to resist combatant propaganda. In particular, they argue that combatants (here, primarily the state) try to manipulate perceptions of the harm they inflict by denying it or blaming it on their opponents, and are often successful in doing so. However, these efforts can be resisted by those with sufficient “ideological capital” that makes them skeptical of the manipulation.

While this is an interesting theory, it isn’t really well tested by the analysis. The results indicate that the impact of violence on voting depends on prewar leftist mobilization (proxied by roads). There is no measurement of state propaganda about the conflict and people’s belief in it, so it is hard to know whether that is really what’s driving the observed effects. In contrast, it could just be an ideological story in which those with a more leftist worldview judge the state’s intentions as more hostile and punish it more for harm inflicted (a la Lyall, Blair, and Imai 2013). In other words, it’s hard to know whether this is due to propaganda and its spread at all.

So where does this leave us? Well, if the authors can test the state propaganda mechanism more directly by looking at, say, survey evidence from the conflict setting, that would be one thing – but, if not, a broader framing that does not lean as much on this one specific mechanism would help. The authors should reframe the argument so that it is just suggesting that ideology shapes the effect of violence on postwar political preferences, building on studies like Lyall, Blair, and Imai 2013 and Silverman 2019 but extending them into the arena of longer-run dynamics and postwar voting behavior. Ideologically-driven resistance to combatant propaganda could be one potential mechanism behind the results, but not the only one – and it shouldn’t be such a central or essential part of the story since it can’t really be directly demonstrated.
\end{displayquote}

\noindent\textbf{Response:} I thank again the reviewer for this thoughtful comment on a key aspect of the manuscript. I completely agree with the reviewer that an alternative mechanism highlighting an ideological reaction to state violence (instead of resistance to propaganda) is completely plausible and more than likely it was present in many communities throughout the country. That said, this mechanism should not be at odds with the propaganda mechanism, and both can be at play in the same community or have different importance depending on the individual.

To account for this, I have changed the theoretical framework section, including a discussion of this mechanism as a plausible alternative that could explain part of the results.
Specifically, the discussion is included in the following two paragraphs (page 10 of the manuscript):

``An alternative but complementary mechanism also explains the differences between communities without accounting for the role of state propaganda.
Previous research shows that the short-term effect of wartime violence on combatant support varies depending on whether the perpetrator is part of the in-group or not \citep{Lyall:2013aa}, as civilians evaluate wartime events according to their previous political orientation \citep{Silverman:2019aa, Pechenkina:2020ul}.
A left-leaning ideological context would determine the way local communities judge wartime events and assign blame.
Thus, even without taking into account propaganda and co-optation strategies by the state, areas that had been more exposed to prewar leftist mobilization should judge wartime state actions more negatively.
Although this mechanism mainly refers to short-term reactions during the conflict, it would not be daring to assume that these beliefs determined political preferences in the long run.

This mechanism hinges on an ideology-driven active judgment of state violence, but it is not at odds with the former discussion on the capacity of some communities to resist state propaganda, which is also driven by local ideology.
During the civil war, information about wartime events in Guatemala was likely very limited, so beliefs about wartime events were formed in a context of high uncertainty.
Along these lines, recent research suggests that rumor evaluation in wartime contexts is largely determined by surrounding social networks \citep{Schon:2021wf}.
Active judgment of wartime events is thus very much intertwined with the `narrative contests' that unfold in a situation where state actors are actively trying to spread misinformation about what happened during the war.''

In addition, as I explain in the response to the next comment, I have included a new section with extensive qualitative evidence supporting the theoretical mechanisms (including the role of propaganda). All in all, I hope the manuscript benefits from these changes, which acknowledge the importance of an alternative mechanism and at the same time introduce qualitative evidence for each of the steps of the mechanism.

\vspace{15pt}
\noindent\textbf{Comment 3:}
\begin{displayquote}
2) Mechanisms: relatedly, it’s hard to pin down the mechanism here more broadly, because there isn’t much direct evidence about many of the links in the rich causal story that is told – prewar social mobilization (proxied by roads, as noted in the piece), leftist ideological penetration (besides the voting behavior DV), and the state propaganda dimension (as discussed above). There is really a desperate need for rich qualitative evidence here. I think an “identifying the mechanisms” section after the main results would help. Can it be qualitatively shown or at least strongly suggested that liberal priests and activists spread via the road network? Can it be shown that the areas they reached then became sites of leftist agitation? Is there evidence of political activism after the war in these areas to commemorate the violence, define it ideologically, etc.? If there is any available survey evidence from one of the Latin American regional survey projects that could speak to the causal links in these chains (conflict attitudes would be ideal of course, but even leftist ideology, distrust of state media, etc. would be helpful).
\end{displayquote}

\noindent\textbf{Response:} I thank the reviewer for this insightful comment, which points to a crucial limitation of the first version of the manuscript, raised also by the other reviewers.
The current version of the manuscript includes a new section (`Identifying the mechanism', pages 24--28), where I try to offer qualitative evidence supporting each step in the proposed mechanism, including the road accessibility assumption.

As I explain in the text (page 24), ``I discuss four main points: how the road infrastructure was a crucial factor explaining the diffusion of political mobilization in the years before the violence, the relationship between the presence of priests and activists and local mobilization activities, the role of state propaganda, and the existence of local commemoration activities in the postwar period.'' I use both archival sources and secondary qualitative evidence, mainly from Sociology and Anthropology.

I hope this new version is more convincing and the mechanism section helps to improve the credibility of the theoretical framework and the quantitative results.
I have included references to this section throughout the main text and changed slightly the theoretical framework to account for the lack of direct evidence for the mechanism in the quantitative analyses.
Regarding survey data, I could not find suitable time series for the variables of interest.

\vspace{15pt}
\noindent\textbf{Comment 4:}
\begin{displayquote}
Other issues:

\begin{itemize}
\item[-] economic development: can you control for economic development across different municipalities? It is possible that with your measure of unpaved roads you’re picking up something like this. It could be that poorer people (and/or poorer areas) are more vulnerable to state coercion in this case, as has been shown elsewhere for example in the electoral violence literature.
\end{itemize}
\end{displayquote}

\noindent\textbf{Response:} This is a thoughtful point about potential omitted variable bias, which also resonates with the other reviewers' comments. Income is indeed an important variable, but I could not find any local-level data, at least for the period before the victimization campaign. Yet, I do include now a measure of local literacy in 1973, which should be a relatively good proxy for income and, in general, capture general levels of local development.

The comments about a potentially stronger effect of state coercion on poorer people could be very relevant. On a more superficial level, I try to control for this by including literacy rate (along with other variables that are more or less related to economic development, such as distance to the capital city and the share of Indigenous population), but it could be argued that poorer people are more exposed, in general, to state repression beyond fatal victimization. However, even though it is an interesting question, it falls outside the scope of this manuscript, particularly because the absence of available data on non-fatal repression---which also includes economic reprisals---would complicate finding a suitable design.

\vspace{15pt}
\noindent\textbf{Comment 5:}
\begin{displayquote}
\begin{itemize}
\item[-] placebo test: can you get data on the leftist vote across areas from the prior democratic period of 1944-54? Since you’re trying to measure the effect of the liberalization through the road network that occurred in the 1960s-70s, controlling for this shouldn’t impact your results and would show that they weren’t due to the areas near the roads having already been more leftist in outlook.
\end{itemize}
\end{displayquote}

\noindent\textbf{Response:} I thank the reviewer for this suggestion, which would greatly improve the empirical design. However, to the best of my knowledge, there is no data on voting patterns during the 1944--1954 period, at least at a disaggreated level that would allow for meaningful analyses.

\vspace{15pt}
\noindent\textbf{Comment 6:}
\begin{displayquote}
\begin{itemize}
\item[-] other parties: what about voting for other parties in the 1999-2015 elections? I noticed that the 2 parties in question weren’t very popular, especially after 1999. So you’re really looking at a very small slice of the vote and trying to predict it. Can the other parties not be sorted ideologically in a way that would allow them to be included in the analysis? And relatedly, if leftist areas were so effective at creating collective memories of victimization which led them to support URNG, how come this support dissipated so rapidly after the first election? This should be at least addressed somewhere in the piece.
\end{itemize}
\end{displayquote}

\noindent\textbf{Response:} This is a very thoughtful comment, which is also related to Comment 3 by R3. The reviewer is right when saying that the loss of votes to both the URNG and the FRG can hide some patterns in the data. My strategy to deal with this was to include in Appendix~\ref{app:results_full} results of the main models but including more parties, both leftist and rightist, in the calculation of the dependent variable, together with the URNG and FRG.
The goal is to try to capture votes that, even if preference-wise are determined by leftist or rightist preferences as a result of wartime dynamics, go to different but related parties.
In particular, I include the \textit{Partido Patriota} (PP) and the \textit{Frente de Convergencia Nacional} (FCN) together with the FRG, and the \textit{Unidad Nacional de la Esperanza} (UNE) together with the URNG.
Both PP and FCN are major parties (most voted parties in some elections, their leaders are or have been Presidents of Guatemala) that have been allegedly linked to a similar discourse as the FRG's and to military officers accused of human rights violations during the war.
The UNE is also a major left-wing party that has headed by former President of Guatemala Álvaro Colom, who had been a member of URNG before.
In any case, one of the reasons why the effect disappears might be because after 2007 there was barely any discourse among left-wing parties that directly pointed to the revolutionary spirit of the guerrillas \citep{Ibarra:2008to}.

\vspace{15pt}
\noindent\textbf{Comment 7:}
\begin{displayquote}
\begin{itemize}
\item[-] time decay: related to the last point, we can also see in the Appendix that the effects are pretty robust but in many cases (e.g. Figure A3) seem to fade significantly with time. Is this evidence that supports the argument, since it shows that they are strongest where we’d expect – right after the war? And later on voting happens more for other reasons? Or does it get at the weakness and limited duration of the results? Again, this should at least be engaged with somewhere, possibly in the conclusion.
\end{itemize}
\end{displayquote}

\noindent\textbf{Response:} I thank the reviewer for this comment. I completely agree that this is an important issue that was not addressed properly in the previous version of the manuscript. I have now included a short discussion at the end of the Results section (on pages 23--24) arguing that this decline of importance over time is not at odds with the argument:

``In Appendix ~\ref{app:results_year} I also include results using a cross-section of the data for each election.
These results show that the effects of victimization was mainly present in the 1999 elections, and waned after that year.
In principle, this is coherent with the argument, since the difference between municipalities exposed to prewar mobilization and the rest should be, if anything, stronger right after the war and become less relevant after democracy was restored, for at least two reasons.
First, new issues different from wartime events emerged and increasingly defined voting patterns.
And second, new possibilities for political mobilization opened up, blurring the effect of the mechanisms discussed here.''

\newpage
\section*{Reviewer 3}

I would like to thank the reviewer for all the comments. They were all very insightful and I found them very helpful in improving the manuscript. I respond to each one below.

\vspace{15pt}
\noindent\textbf{Comment 1:}
\begin{displayquote}
Below are a few things for the author to consider:
What about corruption? This is not mentioned once in the article. Recent events, including the expulsion of CICIG by the Jimmy Morales administration, demonstrate this troubling trend. I would argue that it is important to understand the notion of state fragility that has plagued Guatemala over decades, including the period analyzed in this article. Adriana Beltrán and other scholars have written about corruption and impunity. Some surveys (e.g., LAPOP) ask about perceptions of corruption and bribes from different government actors.
\end{displayquote}

\noindent\textbf{Response:} I thank the reviewer for raising this point, which speaks to a very important aspect of contemporary Guatemalan politics. I now include a mention (page 7) to this issue, including some references that could be useful for the reader to further explore this topic. However, because of space constraints, I cannot engage in a longer discussion of corruption in Guatemala. I agree with the reviewer that it is crucial when focusing on Guatemala, and some authors have indeed linked the legacies of the civil war to postwar state corruption \citep{Peacock:2003tt} or other types of violent institutions with a certain degree of connection with state authorities \citep{Bateson:2013aa}. Yet, even though I do agree that they are very relevant, I think that a fully developed account of these factors would merit a different study and would indeed open an interesting research agenda.

\vspace{15pt}
\noindent\textbf{Comment 2:}
\begin{displayquote}
While this is not the focus of this article, it is worth mentioning, even if in a footnote, that Guatemala has an intricate relationship between the state, gangs (MS-13 and the 18th Street), and organized crime (see the work of José Miguel Cruz). Scholars like Steven Dudley have referred to Guatemala as a mafia state. There are zones in Guatemala (“red zones”) that are controlled by street gangs.
\end{displayquote}

\noindent\textbf{Response:} I thank the reviewer for this comment. Similar to the previous point about corruption, organized crime is a very important aspect of Guatemalan politics. I now mention it in the main text, particularly referencing the work by \citet{Levenson:2013tm}, who discusses the political dimension of the emergence of youth gangs. I cannot include a full discussion of this issue because of space constraints, even though I agree that exploring this could be a very fruitful research agenda. Indeed, previous works link the historical legacies of conflicts to mobilization against organized crime in other Latin American countries \citep{Osorio:2021aa}. I believe this is a very interesting research question, but it falls outside the scope of this manuscript.

\vspace{15pt}
\noindent\textbf{Comment 3:}
\begin{displayquote}
One may question whether people vote for parties or leaders. In Central America, politicians have formed their own political parties (e.g., Bukele and the New Ideas party). There have been dozens of parties in Guatemala during the political election cycle. Today, for instance, there are 28 parties registered in Guatemala.
\end{displayquote}

\noindent\textbf{Response:} I thank the reviewer for this insightful comment, which is also related to comment 6 by R2. I agree that the analyses would benefit if more parties were included in the calculation of the dependent variable since, as the reviewer affirms, partisan voting might not be as important in Guatemala.

In order to address this issue, I now include additional analyses in Appendix~\ref{app:results_full} including more parties, both leftist and rightist, to calculate the dependent variable.
The goal is to include related parties that could be capturing a similar vote preference---support for the former military regime or for the former rebels---as the FRG and the URNG.
In particular, I include the \textit{Partido Patriota} (PP) and the \textit{Frente de Convergencia Nacional} (FCN) together with the FRG, and the \textit{Unidad Nacional de la Esperanza} (UNE) together with the URNG.
I hope these changes help to alleviate the concern about non-partisan voting patterns.

\vspace{15pt}
\noindent\textbf{Comment 4:}
\begin{displayquote}
This article does not mention the issue of the police once. Does police corruption and ungoverned spaces impact your analysis? Perhaps this is worth mentioning, even if in a footnote.
\end{displayquote}

\noindent\textbf{Response:} I agree with the reviewer that this is an important issue when studying Guatemala. Even though I cannot engage in a lengthy discussion because of space constraints, I do now mention these issues---along with state corruption and organized crime---in a footnote (page 7), and refer the reader to some related references.

\vspace{15pt}
\noindent\textbf{Comment 5:}
\begin{displayquote}
The author uses pooled OLS regressions. A reader might want to know if the author tested for heteroskedasticity or multicollinearity using the Variance Inflation Factor (VIF).
\end{displayquote}

\noindent\textbf{Response:} I thank the reviewer for this comment, which resonates with R1's comment about a potential concern for the inflation of standard errors. To have a better idea about the correlation among the covariates, I include in Appendix~\ref{app:stats} a correlation plot (Figure~\ref{fig:corrplot}) showing that none of the variables is strongly correlated with one another. One possible exception is the \% of Indigenous population, which has a strong correlation with the local literacy rate. However, the latter variable was not included in the previous version of the manuscript, and results have not changed significantly after its inclusion in the current version.
Moreover, following R1's recommendation, I have included models without the control variables for some of the key sections, particularly in Appendix~\ref{app:lm_violence}, where the relationship between state violence and road accessibility is tested.

\vspace{15pt}
\noindent\textbf{Comment 6:}
\begin{displayquote}
In your pooled OLS model, a reader may wonder why the author did not include several variables (e.g., income, education, corruption measures, and trust in institutions like the military or police). This could be something worth addressing, even if in a footnote.
\end{displayquote}

\noindent\textbf{Response:} Thank you for this comment. I agree that omission of some of these variables might be seen as odd. The main reason I did not include them is that I could not find them at the level of municipalities and particularly for the period before the war. However, I now include in all models the literacy rate, drawn from the 1973 census, which should be a good proxy for income and economic development in general. I hope the inclusion of this variable helps partially solving these omissions.

\vspace{15pt}
\noindent\textbf{Comment 7:}
\begin{displayquote}
I would also recommend citing/ reviewing the work of Deborah T. Levenson, Anthony W. Fontes, Adriana Beltrán, Christine Wade, and José Miguel Cruz.
\end{displayquote}

\noindent\textbf{Response:} I now include some of these works, particularly the ones related to the role of organized crime and youth gangs in Guatemala, such as \citet{Peacock:2003tt, Beltran:2016td, Booth:2010wd, Levenson:2013tm}.

\newpage
\bibliographystyle{jpr}
\bibliography{REF}

\end{document}
