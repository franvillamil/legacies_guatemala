\documentclass[12pt, a4paper, notitlepage]{article}
\usepackage[margin=2.25cm]{geometry}
\usepackage[english]{babel}
\usepackage[utf8]{inputenc}
\usepackage{csquotes, xcolor}
\usepackage{setspace}
\setstretch{1.3}

% Bibliography
\usepackage[round]{natbib}
\setlength{\bibsep}{5pt}

\usepackage{xr-hyper}
\usepackage[colorlinks = TRUE, allcolors = blue]{hyperref}
\externaldocument{main}
\externaldocument{appendix}

\renewcommand*\rmdefault{ppl}

\usepackage[]{titlesec}
\titleformat*{\section}{\large\bfseries}

\title{\large \textbf{Revision memo}\\{\normalsize Manuscript `Violence, co-optation, and postwar voting in Guatemala', submitted to Conflict Management and Peace Science (CMPS-21-0042.R1)}}
\author{}
\date{\today}

\definecolor{darkgrey}{rgb}{0.5, 0.5, 0.5}
\renewcommand{\mkbegdispquote}[2]{\color{darkgray}}

\begin{document}

\maketitle

% \newpage
\section*{Response to Editor}

\newpage
\section*{Reviewer 4}

First, I would like to thank the reviewer for all the comments. They were very thoughtful and useful in improving the manuscript, particularly in terms of making the argument and the contribution more clear. I respond to each one below.

\vspace{15pt}
\noindent\textbf{Comment 1:}
\begin{displayquote}
It is unclear whether the author is focusing on violence or propaganda-while these two things are closely related they are also very different. More clarity on this is needed throughout.
\end{displayquote}

\noindent\textbf{Response:}

I focus on the effect of violence and how it changes depending on exposure to propaganda.

But true it is not clear. Have rewritten a few parts of the manuscript to make it more clear. Especially the introduction, including also suggestion made in comment 3 below:

``How do these strategies alter the effect of wartime victimization on political preferences after the war?
The theoretical framework found in previous research is not able to answer this question, as past works do not usually account for any local factor that could mediate the long-term effect of civilian victimization.
Individuals are assumed to objectively interpret violent events, and no attention is paid to those external factors that might influence this interpretation, such as propaganda or co-optation efforts by the perpetrator.

This paper tries to address this limitation, studying the effect of violence on postwar preferences and how this effect can vary depending on the exposure and reactions to propaganda and co-optation strategies.''

\vspace{15pt}
\noindent\textbf{Comment 2:}
\begin{displayquote}
Author states previous work treats relationship between violence and civilian political preferences as a black box - can they expand? What information is missing from previous research?
\end{displayquote}

\noindent\textbf{Response:}

This comment hints at the idea that the effect of violence is usually theoretized to take place homogenously, without discussion about potential third-factors that influence its present

Added ``Previous research has treated the process that leads from violent events to a change in political preferences as a black box.
In other words, we know little about potential heterogenous effects or the factors that explain why legacies of violence might be present in some cases but not in others.''


\vspace{15pt}
\noindent\textbf{Comment 3:}
\begin{displayquote}
"Individuals are assumed to objectively interpret violent events, and no attention is paid to those external factors that might influence this interpretation, such as propaganda or cooptation efforts by the perpetrator". -- this sentence is key-would put it in the introduction to clarify contribution from the beginning.
\end{displayquote}

\noindent\textbf{Response:}

Moved to the introduction, related to attempt to make more clear the RQ of the manuscript.

\vspace{15pt}
\noindent\textbf{Comment 4:}
\begin{displayquote}
"I focus on the case of Guatemala, which is a good example of the use of denying and cooptation strategies". - does this mean it's unique? How does it compare to other cases?
\end{displayquote}

\noindent\textbf{Response:}

Justify Guate case

Have rephrased in the introduction to:
``I focus on the case of Guatemala, where there was an active debate about the responsibility of wartime events and where state authorities actively engaged in cooptation strategies as part of the counterinsurgency campaign.
In the aftermath of the victimization campaign that took place in Guatemala in the early 1980s, the state tried to deny war events, set up a discourse that justified military actions as a necessary step to bring peace to the country, and forcefully recruited civilians into paramilitary units of local defense.
This example is far from unique, as in many postwar contexts there is a controversy about blame attribution, where active political actors take an active role in trying to define the mainstream discourse.
The fact that a civil war is usually a high-uncertainty context makes it an ideal space for rumors and propaganda to spread \citep{Schon:2021wf}, a situation that in many cases strongly defines the postwar context.''

\vspace{15pt}
\noindent\textbf{Comment 5:}
\begin{displayquote}
How does the article relate to research on misinformation? Are there parallels to be drawn from the use of misinformation in political campaigns/where it is most successful?
\end{displayquote}

\noindent\textbf{Response:}

A bit difficult to draw a parallelism btw the misinfo lit, which focuses mainly in the US (or other developed democracies), and a context of armed conflict in an underdeveloped country.

In any case, I agree it is important to refer to them. Even tho word limit constrains do not allow a lengthy discussion, I did mention that the formation of collective memories might be related to what the misinfo lit refer to as rumors, which is a different phenomenon than misinformation. If misinfo is referred to as blah blah, rumors are, according to \citet{Berinsky:2017ty}, blah blah


\vspace{15pt}
\noindent\textbf{Comment 6:}
\begin{displayquote}
I'm unclear on how a scorched earth campaign could gain support from the civilian population
\end{displayquote}

\noindent\textbf{Response:}

Agree with the reviewer that this is a counterintruitive finding, but highlighting this heterogenous effects of violence is precisely the goal of the manuscript.

Paragraph 2 in page 2 summarizes how this could have happened, mainly through fear:

``In the aftermath of the victimization campaign that took place in the early 1980s, the Guatemalan state tried to deny war events, set up a discourse that justified military actions as a necessary step to bring peace to the country, and forcefully recruited civilians into paramilitary units of local defense.
In a country where many areas had been isolated from national politics and where political illiteracy was rampant, this strategy was to a large extent successful.''

Similarly, in page 7:

``The victimization campaign of the early 1980s disrupted local communities. It pitted neighbors against neighbors and installed a climate of fear and distrust that would leave long-term consequences \citep{Burrell:2013aa}.
The government successfully destroyed Maya social organizations and gained forced cooperation from local people. (...)

The PAC system was designed to ensure compliance within the countryside.
It forced local civilians to `defend' themselves from the guerrilla and, in many cases, successfully convinced them that they were on the good side and that violence and insecurity was the rebels' fault.''

\vspace{15pt}
\noindent\textbf{Comment 7:}
\begin{displayquote}
With the PAC system-couldn't civilians just go through the motions without actually believing rebels were the enemy? What did the government do to actually convince them?
\end{displayquote}

\noindent\textbf{Response:}

agree with the reviewer it is a possibility and should be mentioned. but anyway that's what the state actually tried etc

changed the text and now it reads as:

``The system aimed to destroy community networks and erode local interpersonal trust, which presumably had helped the guerrillas organize a local base of support \citep{SaenzdeTejada:2004aa}.
Although civilians could in theory take part in the PAC without changing their beliefs, the system was designed with the goal of doing so.
As \citet[641]{Bateson:2017aa} says, ``during the Guatemalan civil war, the military made a concerted effort to socialize and re-educate the civilians of the Western Highlands.''''


\vspace{15pt}
\noindent\textbf{Comment 8:}
\begin{displayquote}
I still don't understand how the state managed to avoid being blamed for acts of wartime victimization. Need more details in the theory and evidence section on how propaganda could overcome experiences with wartime victimization.
\end{displayquote}

\noindent\textbf{Response:}

As with comment 6 above, this counterintuitive finding is actually at the core of the argument. Blah blah. Basically the theoretical argument is the answer: not only that some communities backfired (which is actually what we would expect according to previous research), but that the opposite was true in some other communities.

Also, have added some clarifications at the end of the page 8: refering to the fear and cooptation mechanisms.

End of first subsection of ``Violence, co-optation, and mobilization in Guatemala'':

``I argue that the strategy carried out by state authorities was key to understand the long-term consequences of victimization in Guatemala.
First, this strategy was in many cases successful in establishing a narrative favoring the state's discourse.
In these areas, the state managed to avoid being blamed for acts of wartime victimization, through a fear-based mechanism and the co-optation of local communities, inhibiting the creation of collective memories.
On the contrary, some communities were more capable of resisting state propaganda and building memories of the conflict which led to long-term counter-mobilization as a response to violence.
My argument states that this resistance was possible in places that had been exposed to prewar leftist mobilization, as they had the ideological tools to interpret wartime events differently.``

\vspace{15pt}
\noindent\textbf{Comment 9:}
\begin{displayquote}
The author needs to differentiate between communities' ability to resist co-optation strategies, and just being ideologically opposed to the government. Isn't it more plausible that these communities were simply ideologically opposed to the government? I understand that these two mechanisms could operate in conjunction with one another but I remain unconvinced that the author's plays a significant role.
\end{displayquote}

\noindent\textbf{Response:}


``An alternative mechanism would be that some local communities reacted differently to state violence not because they were less sensitive to propaganda and co-optation but because they had leftist preferences to start with.
Previous research shows that the short-term effect of wartime violence on combatant support varies depending on whether the perpetrator is part of the in-group or not \citep{Lyall:2013aa}, as civilians evaluate wartime events according to their previous political orientation \citep{Silverman:2019aa, Pechenkina:2020ul}.
Left-leaning communities would judge wartime events and assign blame according to their preferences.
Thus, even without taking into account propaganda and co-optation strategies by the state, areas that had been more exposed to prewar leftist mobilization should judged wartime state actions more negatively.
% Although this mechanism mainly refers to short-term reactions during the conflict, it would not be daring to assume that these beliefs determined political preferences in the long run.

This mechanism hinges on an ideology-driven active judgment of state violence, but it is not at odds with the former discussion on the capacity of some communities to resist state propaganda, which is also driven by local ideology.
During the civil war, information about wartime events in Guatemala was likely very limited, so beliefs about wartime events were formed in a context of high uncertainty.
Along these lines, recent research suggests that rumor evaluation in wartime contexts is largely determined by surrounding social networks \citep{Schon:2021wf}.
Active judgment of wartime events is thus very much intertwined with the `narrative contests' that unfold in a situation where state actors are actively trying to spread misinformation about what happened during the war.
The key aspect here is whether propaganda and co-optation had an effect on those communities that did not react against violence.

The quantitative evidence can only offer limited evidence about these mechanisms.
However, in a separate section below, I discuss qualitative evidence in support of the assumptions of the argument and the proxy variables used, including the role of state propaganda.''


\vspace{15pt}
\noindent\textbf{Comment 10:}
\begin{displayquote}
Why is there no data on vote share for the FRG in 2011? Why does vote share drop so significantly for both parties over the years?
\end{displayquote}

\noindent\textbf{Response:}

explain that the FRG did not run in 2011 blah blah (mentioned in footnote 5).
the decline of vote over the years is indeed a more serious question, raised by previous reviewers.

I could not engage in a longer discussion on this in the main text because of space constrains.

From previous revision memo:

``My strategy to deal with this was to include in Appendix F results of the main models but including more parties, both leftist and rightist, in the calculation of the dependent variable, together with the URNG and FRG. The goal is to try to capture votes that, even if preference-wise are determined by leftist or rightist preferences as a result of wartime dynamics, go to different but related parties. In particular, I include the Partido Patriota (PP) and the Frente de Convergencia Nacional (FCN) together with the FRG, and the Unidad Nacional de la Esperanza (UNE) together with the URNG. Both PP and FCN are major parties (most voted parties in some elections, their leaders are or have been Presidents of Guatemala) that have been allegedly linked to a similar discourse as the FRG’s and to military officers accused of human rights violations during the war. The UNE is also a major left-wing party that has headed by former PresidentofGuatemalaA ́lvaroColom,whohadbeenamemberofURNGbefore.Inany case, one of the reasons why the effect disappears might be because after 2007 there was barely any discourse among left-wing parties that directly pointed to the revolutionary spirit of the guerrillas \citep{Ibarra:2008to}.''

\vspace{15pt}
\noindent\textbf{Comment 11:}
\begin{displayquote}
The author blends together a lot of concepts in their explanation of what determines communities' reaction to propaganda-pre-war political mobilization, literacy, collective story telling, ideological capital-it becomes confusing as to which one is the most important. If the author thinks they are all equally important, that should be laid out from the beginning and in the theory section.
\end{displayquote}

\noindent\textbf{Response:}

Have rewritten text, removed some concepts that might confuse (`ideological capital') and focus more on how prewar political mobilizatiom by the left gave these communities blah blah

\vspace{15pt}
\noindent\textbf{Comment 12:}
\begin{displayquote}
The author should explain why they include the controls that they do-as of right now there is no explanation, just a list of variables.
\end{displayquote}

\noindent\textbf{Response:}

Have expanded the section on control variables to include brief explanations for each control. In particular:

``Using data from the 1973 census, obtained from the replication data for \citet{Sullivan:2012aa}, I include the logged population of each municipality, the share of indigenous population, and the literacy rate, which were important drivers of both wartime violence and postwar electoral results.
To control for geographic factors that determined wartime dynamics, I include two measures of terrain ruggedness: the standard deviation of elevation within each municipality \citep{Mapzen:2018aa} and the share of forest cover from the GlobCover Land Cover Maps \citep{Arino:2012aa}.
I include a measure of rebel violence before 1978, the logged number of killings for every 1000 inhabitants following the CIIDH dataset \citep{Ball:1999aa}, to control for early rebel presence.
I also include the logged distance (km) from each municipality to Guatemala City and the logged area of each municipality (in $km^2$), to control for the isolation mechanisms related to both wartime violence and electoral behavior.''

\vspace{15pt}
\noindent\textbf{Comment 13:}
\begin{displayquote}
The paper would benefit from more discussion of the quantitative results-what do they actually mean, what can we conclude from them?
\end{displayquote}

\noindent\textbf{Response:}

End of results section:

``Overall, the results show evidence that violence only backfired in municipalities that were allegedly exposed to prewar leftist mobilization.
The lack of support for the last hypothesis means that it unlikely that state propaganda and co-optation was successful in increasing support as a result of victimization, but this does not mean that it was successful in avoiding counter-mobilization.
As discussed above, the quantitative analyses can offer only limited evidence on the mechanisms.
In the next section below, I offer qualitative evidence on several steps of the mechanism, including the role of state propaganda in preventing victimization from backfiring.''


\vspace{15pt}
\noindent\textbf{Comment 14:}
\begin{displayquote}
The qualititave section seems ad hoc-what strategy did the author use to find these sources? What sources were reviewed?
\end{displayquote}


\noindent\textbf{Response:}

Explain where they come from and what type of tests.

I have added a few lines to the manuscript to explain this:

``The qualitative evidence provided here comes from a mix of sources, including anthropological studies and archives, but does not constitute a strict test of the process.
Rather, it offers a series of non-decisive, `straw in the wind' tests \citep{Collier:2011ve} for several steps of the mechanism and the proxy variables used that cannot be directly confirmed in the quantitative analyses.''

\newpage
\bibliographystyle{jpr}
\bibliography{REF}

\end{document}
